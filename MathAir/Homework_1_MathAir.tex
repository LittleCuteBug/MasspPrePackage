\documentclass[12pt,letterpaper]{article}
\usepackage[utf8]{vietnam}
\usepackage{fullpage}
\usepackage[top=2cm, bottom=4.5cm, left=2.5cm, right=2.5cm]{geometry}
\usepackage{amsmath,amsthm,amsfonts,amssymb,amscd}
\usepackage{lastpage}
\usepackage{enumerate}
\usepackage{fancyhdr}
\usepackage{mathrsfs}
\usepackage{xcolor}
\usepackage{graphicx}
\usepackage{listings}
\usepackage{hyperref}

\hypersetup{%
  colorlinks=true,
  linkcolor=blue,
  linkbordercolor={0 0 1}
}
 
\renewcommand\lstlistingname{Algorithm}
\renewcommand\lstlistlistingname{Algorithms}
\def\lstlistingautorefname{Alg.}

\lstdefinestyle{Python}{
    language        = Python,
    frame           = lines, 
    basicstyle      = \footnotesize,
    keywordstyle    = \color{blue},
    stringstyle     = \color{green},
    commentstyle    = \color{red}\ttfamily
}

\setlength{\parindent}{0.0in}
\setlength{\parskip}{0.05in}

% Edit these as appropriate
\newcommand\course{MathAir2018}
\newcommand\hwnumber{Session 1}                  % <-- homework number
\newcommand\NetIDa{Nguyen Quang Huy}           % <-- NetID of person #1

\pagestyle{fancyplain}
\headheight 35pt
\lhead{\NetIDa}
%\lhead{\NetIDa\\\NetIDb}                 % <-- Comment this line out for problem sets (make sure you are person #1)
\chead{\textbf{\Large Homework \hwnumber}}
\rhead{\course \\ \today}
\lfoot{}
\cfoot{}
\rfoot{\small\thepage}
\headsep 1.5em

\begin{document}

\section*{Problem 1}

Trí tuệ là gì?

\begin{enumerate}
    \item
    Trí tuệ là khả năng nhận thức, khả năng xử lý thông tin và có thể phát triển thông qua rèn luyện và học hỏi, trí tuệ có thể tóm gọn bởi 2 khái niệm học và áp dụng kiến thức.
    \item
    Có nhiều loại trí tuệ ở con người được chia theo các lĩnh vực như khả năng sử dụng ngôn ngữ, khả năng logic, khả năng cảm thụ và sáng tác âm nhạc, khả năng định hướng và xác định vị trí...
    Những khả năng đó đều đòi hỏi con người hay động vật tiếp nhận thông tin (như nghe/ nhìn/ xác định vấn đề...) và xử lý thông tin đó, những khả năng có thể là bẩm sinh hay được học hỏi, chia sẻ, được phát triển trong quá trình cá nhân hay cá thể trưởng thành và có thể mai một khi không sử dụng.
    \item
    Trong ngành Machine Learning, trí thông minh còn gồm có khả năng suy diễn từ kết quả ra nguyên nhân, khả năng lên kế hoạch, khả năng sáng tạo, khả năng trực quan, khả năng tưởng tượng, khả năng tự biết được các kiến thức cơ bản.
\end{enumerate}


\section*{Problem 2}
Trí tuệ nhân tạo là gì? Những tiến bộ nào góp phần dẫn đến cuộc cách mạng AI?
\begin{enumerate}
    \item 
    Trí tuệ nhân tạo là trí tuệ do con người tạo ra có khả năng thực hiện việc học hỏi và nhận thức, áp dụng kiến thức được học để giúp tăng cường trí tuệ của con người hoặc thay thế con người làm việc hoặc nghiên cứu.
    \item
    Như vậy, trí tuệ nhân tạo là máy tính với một số chức năng về trí tuệ giống với con người.
    \item
    Những tiến bộ góp phần dẫn đến cuộc cách mạng của AI:
    \begin{enumerate}
        \item 
        Sự phát minh ra các công cụ toán học, đại số, xác suất thống kê.
        \item
        Phát minh ra các giải thuật học máy.
        \item 
        Sự phát triển công nghệ cấp số mũ.
        \item 
        Con người tạo ra nguồn dữ liệu khổng lồ tạo tiềm năng để khai thác dữ liệu.
        \item
        Cuộc sống con người tiến bộ hơn đòi hỏi cần xử dụng trí tuệ nhân tạo trong nhiều lĩnh vực và ngành nghề.
    \end{enumerate}
\end{enumerate}
\section*{Problem 3}
Học là gì? Học máy là gì? Vị trí của Machine Learning trong AI, Kiến thức kỹ năng có thể được biểu diễn trong máy tính ra sao?
\begin{enumerate}
    \item
    Học hay còn gọi là học tập, học hành, học hỏi là quá trình tiếp thu cái mới hoặc bổ sung, trau dồi các kiến thức, kỹ năng, kinh nghiệm, giá trị, nhận thức hoặc sở thích từ thầy cô, bạn bè và có thể liên quan đến việc tổng hợp các loại thông tin khác nhau. (Định nghĩa từ Wikipedia)
    \item
    Học máy là thuật toán để tìm ra hàm ẩn tối ưu từ những dữ liệu. Máy tính tự động học qua các trải nghiệm.
    \item
    Machine Learning là một phần nhỏ/ một ngành nhỏ của AI, có khả năng tự học mà không phải lập trình một cách rõ ràng, hay có thể hiểu là tự lập trình qua các trải nghiệm.
    \item
    Kiến thức và kĩ năng có thể biểu diễn trong bài toán bằng cách mô hình hoá bài toán, từ đó tìm ra các hàm số thần kì trong không gian các hàm số xấp xỉ.
\end{enumerate}
\section*{Problem 4}
Các thành phần cơ bản (TEFPA như trong bài giảng) cần được mô tả và cung cấp để máy tính tự học cách giải quyết một tác vụ là gì?
\\
TEFPA bao gồm có Task (T), Experience (E), Function Space (F), Performance Measure (P), Search Algorithm (A), trong đó:
\begin{enumerate}
    \item Task: là cho bài toán, đặt ra nhiệm vụ của bài toán.
    \item Experience: là những trải nghiệm cho máy tính, ví dụ như cách sử dụng bộ dữ liệu, training data để huấn luyện máy tính.
    \item Function Space dùng để mô hình hoá bài toán, Function Space là không gian hàm số, giúp khái quát hoá hàm số cần tìm.
    \item Performance Measure là chuẩn đánh giá để biết độ tốt xấu của hàm số tìm được trong Function Space.
    \item Search Algorithm là thuật toán để tìm ra hàm số tối ưu trong không gian hàm số (Function Space)
\end{enumerate}
\section* {Problem 5}
Mô tả tác vụ (đầu vào và đầu ra là gì), kinh nghiệm/dữ liệu cần chuẩn bị (cần thu thập dữ liệu gì, cách thu thập ra sao, trong bao lâu, cần ai giúp đỡ việc gì, mức độ khó khăn về thời gian công sức trang thiết bị v.v.), cách thức đánh giá chất lượng đầu ra cho các ứng dụng:
\begin{enumerate}
    \item Chuyển ảnh chất lượng kém thành ảnh rõ nét
    \item Máy in xử lý ảnh chụp X-quang và chuẩn đoán bệnh
    \item Máy tính đọc một email của khách hàng và tự chuyển đến thư mục tương ứng như cảm ơn, khiếu nại, hỏi thông tin, xin việc,..
\end{enumerate}

\textbf {Chuyển hình ảnh chất lượng kém thành rõ nét:}
\begin{enumerate}
    \item Tác vụ:
    \begin{enumerate}
        \item Đầu vào: ảnh chất lượng thấp.
        \item Đầu ra: ảnh chất lượng cao đã được chỉnh sửa.
    \end{enumerate}
    \item Kinh nghiệm, dữ liệu cần chuẩn bị:
    \begin{enumerate}
        \item Dữ liệu: các cặp ảnh có chất lượng thấp và cao.
        \item Cách thu thập: xử dụng các ảnh đen trắng hoặc ảnh màu có chất lượng cao, downside thành ảnh đen trắng hoặc ảnh màu có chất lượng thấp hoặc làm mờ, làm vỡ v.v. Các ảnh có thể lấy từ các thư viện ảnh như Facebook, stock.adobe v.v.
        \item Thời gian: thời gian thu thập lâu vì cần lượng dữ liệu lớn, càng nhiều dữ liệu càng tốt để thuật toán tối ưu, việc hệ thống thành các cặp ảnh và loại bớt các ảnh chất lượng thấp mất thời gian.
        \item Cần ai giúp đỡ: Những người có kiến thức về chỉnh sửa ảnh v.v.
        \item Mức độ khó khăn về thời gian, công sức trang thiết bị: Thời gian tìm đủ lượng dữ liệu cần thiết lâu, cần có đủ bộ nhớ để lưu trữ dữ liệu do độ lớn của ảnh.
    \end{enumerate}
    \item Cách thức đánh giá chất lượng đầu ra: Đánh giá ảnh đầu ra có tốt bằng ảnh chất lượng cao hay không, độ sai khác có nhiều hay không. Có thể đánh giá bằng cách so sánh từng pixel.
\end{enumerate}


\textbf{Máy tính xử lý ảnh chụp X-quang và dự đoán bệnh:}
\begin{enumerate}
    \item Tác vụ: 
    \begin{enumerate}
        \item Đầu vào: ảnh chụp X-quang bộ phận cần chuẩn đoán bệnh.
        \item Đầu ra: chỉ ra nội dung của ảnh chụp X-quang dự đoán bệnh phù hợp với ảnh X-quang.
    \end{enumerate}
    \item Kinh nghiệm, dữ liệu cần chuẩn bị:
    \begin{enumerate}
        \item Dữ liệu: ảnh chụp X-quang bộ phần cần chuẩn đoán bệnh và dự đoán bệnh của bác sĩ, dữ liệu các loại bệnh lý. 
        \item Cách thu thập: Trích suất ảnh chụp X-quang và chuẩn đoán của bác sĩ từ hồ sơ bệnh nhân của các bệnh viện.
        \item Thời gian: Thời gian lâu vì cần sự đa dạng các loại bệnh, ngoài ra nguồn dữ liệu cũng cần lấy từ nhiều bệnh viện. 
        \item Cần ai giúp đỡ: Cần bác sĩ có chuyên môn để đánh giá độ hiệu quả, cố vấn cho kiến thức về các loại bệnh.
        \item Mức độ khó khăn về thời gian, công sức trang thiết bị: Cần thời gian tìm hiểu về các loại bệnh lý, tổ chức lại bộ dữ liệu để đồng bộ dự đoán bệnh của bác sĩ theo một chuẩn chung để tiện cho việc đánh giá.
    \end{enumerate}
    \item Cách thức đánh giá chất lượng đầu ra: độ khớp giữa chuẩn đoán của máy tính và chuẩn đoán của bác sĩ: có bệnh hay không có bệnh, có cùng nhóm bệnh không, có cùng loại bệnh không.
\end{enumerate}

\textbf{Máy tính đọc email khách hàng:}
\begin{enumerate}
    \item Tác vụ: Đọc và phân loại email khách hàng.
    \begin{enumerate}
        \item Đầu vào: Email được gửi đến.
        \item Đầu ra: Địa chỉ thư mục Email cần được chuyển đến.
    \end{enumerate}
    \item Kinh nghiệm, dữ liệu cần chuẩn bị:
    \begin{enumerate}
        \item Dữ liệu: Email và tên loại Email.
        \item Cách thu thập: từ tình nguyện viên chia sẻ các Email của khách hàng.
        \item Thời gian: Lâu vì quá trình thu thập dữ liệu cần nhiều thời gian.
        \item Cần ai giúp đỡ: những người sử dụng email, tình nguyện viên phân loại các email.
        \item Mức độ khó khăn về thời gian, công sức trang thiết bị: Thời gian thu thập dữ liệu lâu có thể dẫn đến cần nhiều kinh phí.
    \end{enumerate}
    \item Cách thức đánh giá chất lượng đầu ra: địa chỉ phân loại mà thuật toán đưa ra có phù hợp với loại thư hay không. 
\end{enumerate}

\section* {Problem 6}
Ý nghĩa câu phát biểu sau: Máy tính "học" bằng cách tìm kiếm trong không gian hàm số (chương trình máy tính)

Dựa vào trải nghiệm, máy tính sẽ tìm kiếm trong không gian hàm số để ra hàm số tối ưu. Hàm số này có thể được ví như kỹ năng mà máy tính đạt được.
\section* {Problem 7}
Hai vấn đề chính mà ta cần đặc biệt chú ý để giúp máy tính tự tìm kiếm hàm có độ khái quát cao là:

Hai vấn đề đó là:
\begin{enumerate}
    \item Độ khái quát của dữ liệu, dữ liệu cần có đủ các trường hợp cũng như là dữ liệu cần có độ lớn.
    \item Không gian hàm số lớn, đủ để bao quát các trường hợp.
\end{enumerate}
\end{document}
